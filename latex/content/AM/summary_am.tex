\section{Summary}
Lagrangian and action integral are defined as follows:
\begin{align}
  L := \sum T - V, \quad S[L] := \int_{t_i}^{t_f} \odif{t} \, L(q_i, \dot{q}_i, t)
\end{align}
we define the Hamiltonian as the Legendre transform of the Lagrangian:
\begin{align}
  H (q_i, p_i, t) := \sum_i p_i \cdot \dot{q}_i - L(q_i, \dot{q}_i, t)
\end{align}
The action integral is stationary under the variation of the path:
\begin{align}
  \fdif{S} & = \delta \int_{t_i}^{t_f} \odif{t} \, L(q_i, \dot{q}_i, t) = \delta \int_{t_i}^{t_f} \odif{t} \, \sum_i p_i \cdot \dot{q}_i - H(q_i, \dot{q}_i, t) = 0
\end{align}
In the \emph{Variational Principle}, we postulate that the motion of a particle is such that the action is stationary, yielding the Euler-Lagrange equation or Hamilton's equations:
\begin{gather}
  \pdv{L}{q_i} - \odv{}{t} \pab{\pdv{L}{\dot{q}_i}}                  = 0                              \\
  \dot{q}_i                                          = \pdv{H}{p_i}, \quad \dot{p}_i = - \pdv{H}{q_i}
\end{gather}
