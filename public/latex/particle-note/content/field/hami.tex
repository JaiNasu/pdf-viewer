\section{Hamilton Formalism}
\subsection{Legendre Transformation}
Now, remember that we obtained the \emph{Hamiltonian} $H$ of the system through the \emph{Legendre transformation} of the Lagrangian $L$:
\begin{align}
  H(q_i, p_i) & = \sum_i p_i \dot{q}_i - L(q_i, \dot{q}_i)
\end{align}
where $i$ represents each degree of freedom of the system.

In the field, the degree of freedom is infinite - each indexed by the spatial position $\vec{x}$, then,
\dfn{Hamiltonian of a Field}{
  The \emph{Hamiltonian} of a field $\psi(\vec{x}, t)$ (whose canonical conjugate field is $\pi(\vec{x}, t)$) is defined as
  \begin{align}
    H[\psi, \pdif{i} \psi, \pi] & = \int \odif[3]{\vec{x}} \, \bab{\pi \cdot \pdif{t} \psi \vphantom{\frac{1}{1}}} - L
    = \int \odif[3]{\vec{x}} \, \bab{\pi \cdot \pdif{t} \psi - \lagr \vphantom{\frac{1}{1}}}
  \end{align}
}
thus, we should define the \emph{Hamiltonian density} $\hami$ as:
\dfn{Hamiltonian Density}{
  \emph{Hamiltonian density} $\hami$ is defined as the Hamiltonian per unit volume of the field:
  \begin{align}
    \hami(\psi, \pdif{i} \psi, \pi, t) & = \pi \cdot \pdif{t} \psi - \lagr(\psi, \pdif{i} \psi, \pdif{t} \psi, t)
  \end{align}
  which satisfies:
  \begin{align}
    \int \odif[3]{\vec{x}} \, \hami(\psi, \pdif{i} \psi, \pi, t) & = H[\psi, \pdif{i} \psi, \pi, t]
  \end{align}
  which makes the Hamiltonian $H$ a functional of the field $\psi$, its spatial derivatives $\pdif{i} \psi$, and the conjugate field $\pi$.
}
\
\nt{
  Similarly to the discrete case, we can write the Lagrangian densty $\lagr$ in terms of the Hamiltonian density $\hami$:
  \begin{align}
    \lagr[\psi(\vec{x}, t), \pi(\vec{x}, t)] & = \pi \cdot \pdif{t} \psi - \hami[\psi(\vec{x}, t), \pi(\vec{x}, t)]
  \end{align}
  where
  \begin{align}
    \pdif{t} \psi & = \pdv{\hami[\psi, \pi]}{\pi}
  \end{align}
}

\subsection{Canonical Equations of Fields}
We are also interested in the \emph{canonical equations} of fields, which are derived from the variational principle:
\begin{align}
  \fdif{S} = 0 \iff \delta \int \odif{t} \int \odif[3]{\vec{x}} \lagr & = \delta \int \odif{t} \int \odif[3]{\vec{x}} \, \pi \cdot \pdif{t} \psi - \hami
\end{align}
The variation on this integral can be expanded as follows:
\begin{align}
  \delta S & = \int \odif{t} \int \odif[3]{\vec{x}} \, \delta \pi \pdif{t} \psi + \pi \delta(\pdif{t} \psi) - \delta \hami                                                                                                       \\
           & = \int \odif{t} \int \odif[3]{\vec{x}} \, \delta \pi \pdif{t} \psi + \pi \pdif{t} (\delta \psi) - \pdv{\hami}{\psi} \delta \psi - \pdv{\hami}{\pi} \delta \pi - \pdv{\hami}{(\pdif{i} \psi)} \delta (\pdif{i} \psi) \\
           & = \int \odif{t} \int \odif[3]{\vec{x}} \pab{\pdif{t} \psi - \pdv{\hami}{\pi}} \delta \pi + \pab{\pi \pdif{t}(\delta \psi) - \pdv{\hami}{\psi} \delta \psi - \pdv{\hami}{(\pdif{i} \psi)} \pdif{i}(\delta\psi)}
\end{align}
the $\delta \psi$ term can be integrated by parts:
\begin{align}
   & \qquad \color{eq1!50!eq2} \int \odif{t} \int \odif[3]{\vec{x}} \, \color{eq1}                                       %
  \pi \pdif{t}(\delta \psi) %
  \color{eq2}
  - \pdv{\hami}{(\pdif{i} \psi)} \pdif{i} (\delta \psi)                                                                  \\
   & = \begin{multlined}[t]
         \color{eq1}
         \int \odif{t} \int \odif[3]{\vec{x}} \, \pdif{t} ( \pi \delta \psi)
         - \int \odif{t} \int \odif[3]{\vec{x}} \, \pdif{t} \pi \cdot \delta \psi \\
         \color{eq2}
         - \int \odif{t} \int \odif[3]{\vec{x}} \, \pdif{i} \pab{\pdv{\hami}{(\pdif{i} \psi)} \delta \psi}
         + \int \odif{t} \int \odif[3]{\vec{x}} \, \pdif{i} \pab{\pdv{\hami}{(\pdif{i} \psi)}} \delta \psi
       \end{multlined}         \\
   & = \begin{multlined}[t]
         \color{eq1}
         \int \odif[3]{\vec{x}} \, \bab{ \pi \delta \psi}_{\text{boundary}}
         - \int \odif{t} \int \odif[3]{\vec{x}} \, \pdif{t} \pi \delta \psi \\
         \color{eq2}
         - \int \odif{t} \int_{\text{boundary}} \odif{S} \, \pdv{\hami}{(\pdif{i} \psi)} \delta \psi
         + \int \odif{t} \int \odif[3]{\vec{x}} \, \pdif{i} \pab{\pdv{\hami}{(\pdif{i} \psi)}} \delta \psi
       \end{multlined}
\end{align}
the boundary terms vanish, so we end up with:
\begin{align}
  \color{eq1!50!eq2}
  \int \odif{t} \int \odif[3]{\vec{x}} \,
  \color{eq1}
  \pi \pdif{t}(\delta \psi)
  \color{eq2}
  - \pdv{\hami}{(\pdif{i} \psi)} \pdif{i} (\delta \psi)
   & = \color{eq1} - \int \odif{t} \int \odif[3]{\vec{x}} \, \pdif{t} \pi \delta \psi
  \color{eq2} + \int \odif{t} \int \odif[3]{\vec{x}} \, \pdif{i} \pab{\pdv{\hami}{(\pdif{i} \psi)}} \delta \psi \\
   & =
  \color{eq1!50!eq2}
  - \int \odif{t} \int \odif[3]{\vec{x}} \,\bab{
    \color{eq1} \pdif{t} \pi
    \color{eq2} - \nabla \cdot \pab{\pdv{\hami}{(\nabla \psi)}}
    \color{eq1!50!eq2}
  } \delta \psi
\end{align}
and thus the variation of the action becomes:
\begin{align}
  \delta S & = \int \odif{t} \int \odif[3]{\vec{x}} \,
  \pab{
    \pdif{t} \psi
    - \pdv{\hami}{\pi}
  } \delta \pi
  - \pab{
    \color{eq1} \pdif{t} \pi
    \color{eq2} - \nabla \cdot \pab{\pdv{\hami}{(\nabla \psi)}}
    \color{\txtcolor}
  } \delta \psi
\end{align}
for the action to be stationary, the integrand must vanish:
\begin{align}
  \begin{dcases}
    \pdif{t} \psi - \pdv{\hami}{\pi}                                               & = 0 \\
    \pdif{t} \pi + \pdv{\hami}{\psi} - \pdif{i} \pab{\pdv{\hami}{(\pdif{i} \psi)}} & = 0
  \end{dcases} \iff
  \begin{dcases}
    \pdv{\psi(\vec{x}, t)}{t} & = \pdv{\hami[\psi, \pi]}{\pi}                                                 \\
    \pdv{\pi(\vec{x}, t)}{t}  & = -\pdv{\hami[\psi, \pi]}{\psi} + \pdif{i} \pab{\pdv{\hami}{(\pdif{i} \psi)}}
  \end{dcases}
\end{align}
Thus we have the \emph{canonical equations of fields}:
\thm{Canonical Equations of Fields}{
  The variational principle in Hamilton formalism leads to the \emph{canonical equations of fields}:
  \begin{align}
    \pdv{\psi(\vec{x}, t)}{t} & = \pdv{\hami(\psi, \pdif{i} \psi, \pi)}{\pi}, \quad \pdv{\pi(\vec{x}, t)}{t} = -\pdv{\hami(\psi, \pdif{i} \psi, \pi)}{\psi}  + \pdif{i} \pab{\pdv{\hami(\psi, \pdif{i} \psi, \pi)}{(\pdif{i} \psi)}}
  \end{align}
}
\cor{Canonical Equations of Fields using $H$}{
  Using Theorem \ref{thm:func_deriv_relation}, the \emph{canonical equations of fields} can be re-written using the Hamiltonian $H$ instead of the Hamiltonian density $\hami$:
  \begin{align}
    \pdv{\psi(\vec{x}, t)}{t} & = \fdv{H[\psi, \pi]}{\pi}, \quad \pdv{\pi(\vec{x}, t)}{t} = -\fdv{H[\psi, \pi]}{\psi}
  \end{align}
}

\subsection{Poisson Bracket}
In the discrete case, the time evolution of a physical quantity $X(q_i, p_i, t)$, $\dot{X}$ can be written as:
\begin{align}
  \dot{X} =  \odv{X}{t} & =\pdv{X}{t} + \sum_i \pab{\pdv{X}{q_i} \dot{q}_i + \pdv{X}{p_i} \dot{p}_i} = \pdv{X}{t} + \pobra{X}{H}
\end{align}

In the continous case, a physical quantity $X$ should be an integral of "density" $\tilde{X}$ over some volume:
\begin{align}
  X & = \int_V \odif[3]{\vec{x'}} \, \tilde{X}(\vec{x'}, t)
\end{align}
and assume that $\tilde{X}$ is a function of the field $\psi(\vec{x}, t)$ and its conjugate field $\pi(\vec{x}, t)$ (which makes $X$ a functional of the fields):
\begin{align}
  \implies \quad X[\psi, \pi, t] & = \int_V \odif[3]{\vec{x'}} \, \tilde{X}(\psi(\vec{x'}, t), \pi(\vec{x'}, t), t)
\end{align}
Then the time evolution of $X$ can be written as:
\begin{align}
  \odv{X[\psi, \pi, t]}{t}
   & = \odv{}{t} \int_V \odif[3]{\vec{x'}} \, \tilde{X}(\psi(\vec{x'}, t), \pi(\vec{x'}, t), t) = \int_V \odif[3]{\vec{x'}} \, \odv{\tilde{X}(\psi(\vec{x'}, t), \pi(\vec{x'}, t), t)}{t} \\
   & = \int_V \odif[3]{\vec{x'}} \, \pdv{\tilde{X}(\psi, \pi, t)}{t}
  + \pdv{\tilde{X}(\psi, \pi, t)}{\psi} \pdif{t} \psi
  + \pdv{\tilde{X}(\psi, \pi, t)}{\pi} \pdif{t} \pi                                                                                                                                       \\
   & = \int_V \odif[3]{\vec{x'}} \, \pdv{\tilde{X}}{t}
  + \int_V \odif[3]{\vec{x'}} \, \pab{\pdv{\tilde{X}}{\psi} \pdv{\hami}{\pi} - \pdv{\tilde{X}}{\pi} \pdv{\hami}{\psi}}
\end{align}
Using Theorem \ref{thm:func_deriv_relation}, the partial derivatives can be replaced with functional derivatives:
\begin{align}
  \pdv{\tilde{X}}{t} & = \fdv{X}{t}, \quad \pdv{\hami}{\psi} = \fdv{H}{\psi}, \quad \pdv{\hami}{\pi} = \fdv{H}{\pi}
\end{align}
Thus, we can write the time evolution of $X$ as:
\begin{align}
  \odv{X[\psi, \pi, t]}{t}
   & = \pdv{X[\psi, \pi, t]}{t} + \int \odif[3]{\vec{x'}} \, \pab{\fdv{X}{\psi} \fdv{H}{\pi} - \fdv{X}{\pi} \fdv{H}{\psi}}
\end{align}
If we were to write the coordinates explicitly,
\begin{align}
  \odv{X[\psi, \pi, t]}{t} & = \pdv{X}{t} +
  \int \odif[3]{\vec{x'}} \, \pab{
    \frac{\delta X[\psi, \pi, t]}{\delta \psi(\vec{x'}, t)} \frac{\delta H[\psi, \pi, t]}{\delta \pi(\vec{x'}, t)}
    - \frac{X[\psi, \pi, t]}{\delta \pi(\vec{x'}, t)} \fdv{H[\psi, \pi, t]}{\psi(\vec{x'}, t)}
  }
\end{align}
Comparing with the discrete case, we can define the \emph{Poisson bracket} of two physical quantities $X$ and $Y$ as:
\dfn{Poisson Bracket of a Field}{
  The \emph{Poisson bracket} of two physical quantities $X[\psi, \pi, t]$ and $Y[\psi, \pi, t]$ is defined as:
  \begin{align}
    \pobra{X}{Y} & = \int \odif[3]{\vec{x'}} \, \pab{\fdv{X}{\psi(\vec{x'}, t)} \fdv{Y}{\pi(\vec{x'}, t)} - \fdv{X}{\pi(\vec{x'}, t)} \fdv{Y}{\psi(\vec{x'}, t)}}
  \end{align}
}
\thm{Time Evolution of a Physical Quantity}{
  The time evolution of a physical quantity $X[\psi, \pi, t]$ can be written as:
  \begin{align}
    \odv{X}{t} & = \pdv{X}{t} + \pobra{X}{H}
  \end{align}
}
\thm{Poisson Brackets of Fields}{
  For $\psi$ and its conjugate field $\pi$, the Poisson bracket satisfies the following properties:
  \begin{align}
    \pobra{\psi(\vec{x}, t)}{\pi(\vec{x'}, t)}  & = \delta^3(\vec{x} - \vec{x'}) \\
    \pobra{\psi(\vec{x}, t)}{\psi(\vec{x'}, t)} & = 0                            \\
    \pobra{\pi(\vec{x}, t)}{\pi(\vec{x'}, t)}   & = 0
  \end{align}
}
