\section{Minkowski Spacetime}
\subsection{Lorentz Transformation}
So first, let us derive the "correct transformation" between inertial frames.
One key observation is an implication of the invariance of the speed of light.

If we were to draw the \emph{space-time diagram} of a light propagating in the $x$ direction, it would look like this:
\begin{figure}[htbp]
  \begin{minipage}{0.45\textwidth}
    \centering
    \begin{tikzpicture}
      \draw[->] (-1,0) -- (5,0) node[right] {$x$};
      \draw[->] (0,-1) -- (0,5) node[above] {$ct$};
      \draw[thick, red, ->] (0,0) -- (5,5) node[midway, above left] {light};
      \draw[thick, blue, ->] (-1,-0.666) -- (5,3) node[midway, below right] {$G$};
    \end{tikzpicture}
    \caption{Space-time diagram of observer $R$.}
    \label{fig:light-propagation1}
  \end{minipage}
  \begin{minipage}{0.45\textwidth}
    \centering
    \begin{tikzpicture}
      \draw[->] (-1,0) -- (5,0) node[right] {$\xi$};
      \draw[->] (0,-1) -- (0,5) node[above] {$c \tau$};
      \draw[thick, red, ->] (0,0) -- (5,5) node[midway, above left] {light};
    \end{tikzpicture}
    \caption{Space-time diagram of a observer $G$ moving at $v$.}
    \label{fig:light-propagation2}
  \end{minipage}
\end{figure}

We expect the "correct" transformation $\rel{\Lambda}$ from $R$ to $G$, to have following properties:
\begin{itemize}
  \item The transformation is linear: the light has to be mapped to light(straight line to straight line).
  \item The transformation must be of the same form for both $G \to R$ and $R \to G$.
        \begin{align}
          \rel{\Lambda}_{G \to R} = \rel{\Lambda}^{-1}_{R \to G}
        \end{align}
\end{itemize}
Now, since the light propagates in the $x$ direction, the distance propagated by the light in the $R$ frame is given by:
\begin{align}
  x = ct \implies (ct)^2 - x^2 = 0
\end{align}
and in the $G$ frame, it is given by:
\begin{align}
  \xi = c\tau \implies (c\tau)^2 - \xi^2 = 0 = (ct)^2 - x^2 \label{eq:light-propagation}
\end{align}
but at time $t$, the origin of $G$ is at position $x = vt$, which means at time $t$, $\xi(t) = 0$:
\begin{align}
  \xi(x, t) & = a_1(x - vt)
\end{align}
and by the same logic, from $G$ to $R$,
\begin{align}
  x(\xi, \tau) & = a_2(\xi + v \tau)
\end{align}
substituting the first equation into the second, we get:
\begin{align}
  x = a_2 (a_1(x - vt) + v \tau) & = a_2 a_1 x - a_2 a_1 v t + a_2 v \tau                                   \\
  \implies \quad \tau            & = \frac{1 - a_1 a_2}{a_2 v} x + a_1 t                                    \\
  c \tau                         & = a_1 ct + a_3 x, \quad \text{where  } a_3 = \frac{c - c a_1 a_2}{a_2 v}
\end{align}
For convinience, we just put the time and space coordinates together so that we can use some linear algebra:
\begin{align}
  \begin{pmatrix}
    ct \\
    x
  \end{pmatrix}, \quad \begin{pmatrix}
                         c \tau \\
                         \xi
                       \end{pmatrix}
\end{align}
since we assume the transformation is linear, we can write it as:
\begin{align}
  \begin{pmatrix}
    c \tau \\
    \xi
  \end{pmatrix} & = \begin{pmatrix}
                      a_1                & a_3 \\[6pt]
                      - \dfrac{v}{c} a_1 & a_1
                    \end{pmatrix} \begin{pmatrix}
                                    c t \\
                                    x
                                  \end{pmatrix}
\end{align}
which means that the determinant and trace of the transformation matrix $\rel{\Lambda}$ is given by:
\begin{align}
  \det(\rel{\Lambda}) & = a_1^2 + \frac{v}{c} a_1 a_3 \\
  \Tr(\rel{\Lambda})  & = 2a_1
\end{align}
and here, we use a known property of matrix determinant and trace:
\thm{Determinant and Trace}{
  If a matrix $M$ has eigenvalues $\rel{\Lambda}_i$, then the determinant and trace of the matrix can be expressed as:
  \begin{align}
    \det(M) & = \prod_{i=1}^{n} \rel{\Lambda}_i \\
    \Tr(M)  & = \sum_{i=1}^{n} \rel{\Lambda}_i
  \end{align}
}
notice that there are two eigenvectors:
\begin{align}
  \begin{pmatrix}
    1 \\
    1
  \end{pmatrix} & \text{ and } \begin{pmatrix}
                                 -1 \\
                                 1
                               \end{pmatrix}
\end{align}
the first eigenvector corresponds to the light propagating in the $x$ direction, and the second eigenvector corresponds to the light propagating in the $-x$ direction (hence making the light speed invariant).

Substituting the matrix, we get
\begin{align}
  \rel{\Lambda}_+ \begin{pmatrix}
                    1 \\
                    1
                  \end{pmatrix} & = \begin{pmatrix}
                                      a_1                & a_3 \\[6pt]
                                      - \dfrac{v}{c} a_1 & a_1
                                    \end{pmatrix} \begin{pmatrix}
                                                    1 \\
                                                    1
                                                  \end{pmatrix}       \\
                                  & = \begin{pmatrix}
                                        a_1 + a_3 \\
                                        \pab{1 - \dfrac{v}{c}} a_1
                                      \end{pmatrix} \\
  \rel{\Lambda}_- \begin{pmatrix}
                    -1 \\
                    1
                  \end{pmatrix} & = \begin{pmatrix}
                                      a_1 - a_3 \\[6pt]
                                      \pab{1 + \dfrac{v}{c}} a_1
                                    \end{pmatrix}
\end{align}
which gives
\begin{align}
  \rel{\Lambda}_+ + \rel{\Lambda}_- & = 2  = 2 a_1 \quad  \rel{\Lambda}_+ \rel{\Lambda}_-   = \pab{1 - \dfrac{v^2}{c^2}} a_1^2 = a_1^2 + \frac{v}{c} a_1 a_3
\end{align}
the second equation becomes
\begin{align}
  a_3 & = - \frac{v}{c} a_1
\end{align}
which makes the transformation matrix:
\begin{align}
  \rel{\Lambda}_{R \to G} & = \begin{pmatrix}
                                a_1              & -\frac{v}{c} a_1 \\[6pt]
                                -\frac{v}{c} a_1 & a_1
                              \end{pmatrix} = a_1 \begin{pmatrix}
                                                    1            & -\frac{v}{c} \\[6pt]
                                                    -\frac{v}{c} & 1
                                                  \end{pmatrix}
\end{align}
and by the second requirement, the inverse transformation is given by:
\begin{align}
  \rel{\Lambda}_{R \to G}^{-1}
   & =\begin{pmatrix}
        1            & \dfrac{v}{c} \\[6pt]
        \dfrac{v}{c} & 1
      \end{pmatrix} a_1^{-1} = \rel{\Lambda}_{G \to R} = a_1 \begin{pmatrix}
                                                               1            & \dfrac{v}{c} \\[6pt]
                                                               \dfrac{v}{c} & 1
                                                             \end{pmatrix}
\end{align}
That means
\begin{align}
  \rel{\Lambda}_{R \to G} \rel{\Lambda}_{R \to G}^{-1}
                       & = a_1^2 \begin{pmatrix}
                                   1             & - \dfrac{v}{c} \\[6pt]
                                   -\dfrac{v}{c} & 1
                                 \end{pmatrix} \begin{pmatrix}
                                                 1            & \dfrac{v}{c} \\[6pt]
                                                 \dfrac{v}{c} & 1
                                               \end{pmatrix} = \begin{pmatrix}
                                                                 1 & 0 \\
                                                                 0 & 1
                                                               \end{pmatrix}                                    \\
                       & = a_1^2\begin{pmatrix}
                                  1 - \dfrac{v^2}{c^2} & 0                    \\[6pt]
                                  0                    & 1 - \dfrac{v^2}{c^2}
                                \end{pmatrix}                                   \\
  \implies \quad a_1^2 & = \frac{1}{1 - \dfrac{v^2}{c^2}}                                                            \\
  \implies \quad a_1   & = \frac{1}{\sqrt{1 - \dfrac{v^2}{c^2}}} \qquad \because \rel{\Lambda}_+ \rel{\Lambda}_- > 0
\end{align}
For convinience, let us define a few constants:
\begin{align}
  \beta & = \frac{v}{c}, \quad \gamma = \frac{1}{\sqrt{1 - \beta^2}}
\end{align}
then $\rel{\Lambda}_{R \to G}$ can be written as:
\begin{align}
  \rel{\Lambda}_{R \to G} & = \gamma \begin{pmatrix}
                                       1      & -\beta \\
                                       -\beta & 1
                                     \end{pmatrix}
\end{align}
This is the \emph{Lorentz transformation} between two inertial frames $R$ and $G$ moving at velocity $v$ relative to each other.
\dfn{Lorentz Tranformation}{
  The Lorentz transformation between two inertial frames $R$ and $G$ moving at velocity $v$ relative to each other is given by:
  \begin{align}
    \begin{pmatrix}
      c \tau \\
      \xi
    \end{pmatrix} & = \rel{\Lambda} \begin{pmatrix}
                                      c t \\
                                      x
                                    \end{pmatrix}, \quad \rel{\Lambda} = \gamma \begin{pmatrix}
                                                                                  1      & -\beta \\
                                                                                  -\beta & 1
                                                                                \end{pmatrix}, \quad \text{where  } \beta = \frac{v}{c}, \gamma = \frac{1}{\sqrt{1 - \beta^2}}
  \end{align}
}
note that
\begin{align}
  \gamma^2 (1 - \beta^2) = 1
\end{align}

\cite{geometry-spacetime}

\subsection{Minkowski Spacetime for (1 + 1)}
In the previous discussion, we used the fact that for light, traveling at speed $c$,
\begin{align}
  (ct)^2 - x^2 = 0
\end{align}
and this relation must hold for all inertial frames, from the invariance of the speed of light: that is,
this quantity must be invariant under the Lorentz transformation.

If we remember that rotations do not change the length of a vector which is essentially an inner product of the vector with itself,
maybe we can write the invariant quantity as an inner product.

So the first logical step would be to define a vector:
\begin{align}
  \rel{x} & = \begin{pmatrix}
                ct \\
                x
              \end{pmatrix}
\end{align}
But taking the inner product of this vector with itself, we get
\begin{align}
  \rel{x} \cdot \rel{x} & = (ct)^2 + x^2
\end{align}
which is not the quantity we expected.
How can we "fix" this?
If we remember the definition of the inner product:
\begin{align}
  \rel{x} \cdot \rel{x} & = \rel{x}^T \rel{x} = \begin{pmatrix}
                                                  ct & x
                                                \end{pmatrix}
  \begin{pmatrix}
    ct \\
    x
  \end{pmatrix} = (ct)^2 + x^2
\end{align}
to have the invariant quantity, we need to have a negative sign in the second term.

One possible solution is to introduce a more general definition of the inner product.
We consider that the usual Euclidean inner product is actually in the form:
\begin{align}
  \rel{x} \cdot \rel{x}^\prime & = \rel{x}^T I \rel{x}^\prime = \begin{pmatrix}
                                                                  ct & x
                                                                \end{pmatrix}
  \begin{pmatrix}
    1 & 0 \\
    0 & 1
  \end{pmatrix}
  \begin{pmatrix}
    ct^\prime \\
    x^\prime
  \end{pmatrix} = c^2 t t^\prime + x x^\prime
\end{align}
and by changing the matrix in the middle $I$ to a different matrix $\rel{M}$, we can define a more general inner product:
\begin{align}
  \rel{x} \cdot \rel{x}^\prime
                  & \iffdef \begin{pmatrix}
                              ct & x
                            \end{pmatrix} \begin{pmatrix}
                                            \rel{M}_{00} & \rel{M}_{01} \\
                                            \rel{M}_{10} & \rel{M}_{11}
                                          \end{pmatrix}                                                    \\
  \begin{pmatrix}
    ct^\prime \\
    x^\prime
  \end{pmatrix} & = \begin{pmatrix}
                      ct & x
                    \end{pmatrix} \begin{pmatrix}
                                    \rel{M}_{00} ct + \rel{M}_{01} x \\
                                    \rel{M}_{10} ct + \rel{M}_{11} x
                                  \end{pmatrix}                                                       \\
                  & = \rel{M}_{00} c^2 t t^\prime+ ct (\rel{M}_{01} x + \rel{M}_{10} x^\prime) + \rel{M}_{11} x x^\prime
\end{align}
by comparison, the matrix we want is:
\begin{align}
  \rel{M} & = \begin{pmatrix}
                1 & 0  \\
                0 & -1
              \end{pmatrix}
\end{align}
and let us define this special matrix as the \emph{Minkowski metric} $\rel{\eta}$:
\dfn{Minkowski Metric in (1 + 1)}{
  The Minkowski metric is defined as:
  \begin{align}
    \rel{\eta} & = \begin{pmatrix}
                     1 & 0  \\
                     0 & -1
                   \end{pmatrix}
  \end{align}
}


