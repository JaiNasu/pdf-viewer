\section{Functional Derivative}
\subsection{Definition}
Consider a quantity $I$ defined as follows:
\begin{align}
  I := \int_A^B \odif{x} \, F(x)
\end{align}
Notice that $I$ is not really a function of $x$, but if you had to say, it is more a "function" of $F$ - may be $F(x) = e^x$, or $F(x) = a x^2 + bx + c$, or, etc.
So, to denote the dependence of $I$ on the function $F$, we write
\begin{align}
  I[f] := \int_A^B \odif{x} \, F(x)
\end{align}
This is called a \emph{(linear) functional}.
Now, imagine that $F$ is a function of $f$, for example, $F[f] = f(x)^2$.
By chain rule, a small change in $F$, denoted as $\fdif{F}$, can be expressed as:
\begin{align}
  \fdif{F}[f] & = F[f + \fdif{f}] - F[f] \\
              & = \pdv{F}{f} \fdif{f}
\end{align}
so, similarly,
\begin{align}
  \fdif{I}[F] & = I[F + \fdif{F}] - I[F]                                              \\
              & = \int_A^B \odif{x} \, \fdif{F}[f]                                    \\
              & = \int_A^B \odif{x} \, \pdv{F}{f} \fdif{f} \label{eq:func_deriv_def1}
\end{align}
Then, the \emph{functional derivative} of $I$ with respect to $f$, $\fdv{I}{f}$, is defined as follows:
\dfn{Functional Derivative}{
  If a function $\phi(x)$ exists, such that
  \begin{align}
    \fdif{I} & = \int_A^B \odif{x} \, \phi(x) \fdif{f}(x),
  \end{align}
  we say that $\phi(x)$ is the \emph{functional derivative} of $I$ with respect to $f$, and denote it as
  \begin{align}
    \fdv{I}{f(x)} := \phi(x) \iff \fdif{I} := \int_B^A \odif{x} \, \fdv{I}{f(x)} \fdif{f}(x). \label{eq:func_deriv_def}
  \end{align}
}
Immidiately, by comparing Eq.(\ref{eq:func_deriv_def}) and Eq.(\ref{eq:func_deriv_def1}), we see the following relation:
\thm{Functional-density relation}{
  For a quantity $I[F]$ and its density $F[f(x)]$, the functional derivative satisfies the following relation:
  \begin{align}
    I & = \int_A^B \odif{x} \, F[f(x)] \quad \implies \quad \fdv{I}{f(x)} = \pdv{F[f(x)]}{f(x)}
  \end{align}
  \label{thm:func_deriv_relation}
}



\subsection{Two function case}
Consider a case where $I$ is the functional of $F$, which is also a functional of $f$ and $g$:
\begin{align}
  I[F[f, g]] & = \int_B^A \odif{x} \, F[f(x), g(x)]
\end{align}
Or more generally, if a function $D(x)$ satisfies the following
Now, let us add some small change of $f$, $\fdif{f}$:
\begin{align}
  I[F[f + \fdif{f}, g]] & = \int_B^A \odif{x} \, F[f(x) + \fdif{f}(x), g(x)]   \\
                        & = \int_B^A \odif{x} \, F[f, g] + \pdv{F}{f} \fdif{f}
\end{align}
and similarly, by adding $\fdif{g}$,
\begin{align}
  I[F[f, g + \fdif{g}]] & = \int_B^A \odif{x} \, F[f(x), g(x) + \fdif{g}(x)]   \\
                        & = \int_B^A \odif{x} \, F[f, g] + \pdv{F}{g} \fdif{g}
\end{align}
Combining these two, we have
\begin{align}
  I[F[f + \fdif{f}, g + \fdif{g}]]                         & = \int_B^A \odif{x} \, F[f(x) + \fdif{f}(x), g(x) + \fdif{g}(x)]           \\
                                                           & = \int_B^A \odif{x} \, F[f, g] + \pdv{F}{f} \fdif{f} + \pdv{F}{g} \fdif{g} \\
  \implies \quad I[F[f+\fdif{f}, g+\fdif{g}]] - I[F[f, g]] & = \int_B^A \odif{x} \, \pdv{F}{f} \fdif{f} + \pdv{F}{g} \fdif{g}
\end{align}
In this case, we just like our normal derivative, we should denote the LHS as:
\begin{align}
  \fdif{I} & := \int_B^A \odif{x} \, \pdv{F}{f} \fdif{f} + \pdv{F}{g} \fdif{g}
\end{align}
or alternatively,
\begin{align}
  \fdif{I} & := \int_B^A \odif{x} \, \fdv{I}{f} \fdif{f} + \fdv{I}{g} \fdif{g}
\end{align}

\subsection{Euler-Lagrange Equation}
For the two function case, especially consider that $g = \odv{f}{x}$, and let us see what happens.
Specifically, let us set that $\fdif{f}(A) = \fdif{f}(B) = 0$.
Then, we have:
\begin{align}
  \fdv{I}{g} \fdif{g} & = \fdv{I}{\odv{f}{x}} \fdif{\odv{f}{x}}
  \intertext{we can change the order of the derivative:}
                      & =\fdv{I}{\odv{f}{x}} \odv{\fdif{f}}{x}
  \intertext{from the differentiation of a product, we have}
                      & = \odv{}{x} \pab{\fdv{I}{\pab{\odv{f}{x}}} \fdif{f}} - \odv{}{x} \fdv{I}{\pab{\odv{f}{x}}} \fdif{f}
\end{align}
Now, subsituting this to the two function case, we get:
\begin{align}
  \fdif{I} & =
  \int_B^A \odif{x} \, \bab{\pab{\fdv{I}{f}
      - \odv{}{x} \fdv{I}{\pab{\odv{f}{x}}}} \fdif{f}
    + \odv{}{x} \pab{\fdv{I}{\pab{\odv{f}{x}}} \fdif{f}}}
\end{align}
the total derivative term is zero, since $\fdif{f}(A) = \fdif{f}(B) = 0$.
Thus, we have
\begin{align}
  \fdif{I} & = \int_B^A \odif{x} \, \pab{\fdv{I}{f}
    - \odv{}{x} \fdv{I}{\pab{\odv{f}{x}}}} \fdif{f}
\end{align}
and since $I = \int_B^A \odif{x} \, F[f(x), g(x)]$, we can say that
\begin{align}
  \pdv{F}{f} & = \fdv{I}{f} , \quad \pdv{F}{g} = \fdv{I}{g}
\end{align}
Then
\begin{align}
  \fdif{I} & = \int_B^A \odif{x} \, \pab{
    \pdv{F}{f} - \odv{}{x} \pdv{F}{\pab{\odv{f}{x}}}
  } \fdif{f} \label{eq:func_deriv_euler_lagrange}
\end{align}
And if we somehow want to find a minimum of $I$, we can set $\fdif{I} = 0$:
\begin{align}
  \implies \quad \pdv{F}{f} - \odv{}{x} \pdv{F}{\pab{\odv{f}{x}}} & = 0
\end{align}
This is called the \emph{Euler-Lagrange equation}.
\thm{Euler-Lagrange Equation}{
  For a functional $I\bab{F(f, \odv{f}{x})}$ to be stationary, ($\fdif{I} = 0$), the \emph{Euler-Lagrange equation} must be satisfied:
  \begin{align}
    \delta I \bab{F(f, \odv{f}{x})} \iff \pdv{F}{f} - \odv{}{x} \pab{\pdv{F}{\pab{\odv{f}{x}}}} & = 0
  \end{align}
}

\subsection{Important Property}
In general, consider that the functional $F$ is a function of $f_1(t), f_2(t), \ldots, f_n(t)$:
\begin{align}
  F[f_1, f_2, \ldots, f_n] \implies \fdif{F} & = \sum_{j=1}^{n} \fdv{F}{f_j(t)} \fdif{f_j(t)}
\end{align}
If $F = f_i$, we expect that
\begin{align}
  \fdv{f_i}{f_i} & = 1 \implies \fdif{f_i} = \sum_{j=1}^{n} \fdv{f_i}{f_j} \fdif{f_j}
  \implies \quad \fdv{f_i}{f_j} = \delta_{ij} \label{eq:func_deriv_kronecker_delta}
\end{align}
Similarly, consider a continous case where $F$ is a function of $f(x)$:,
\begin{align}
  F[f(x,t)] \implies \fdif{F} & = \int_A^B \odif{x'} \, \fdv{F}{f(x',t)} \delta{f(x',t)}
\end{align}
Note the distinction between the variable $x$ and the integration variable $x'$. This is because $x$ is an "index" of $f(t)$: $f_i \to f(x)$.
Then, if we set $F = f(x)$, we expect that
\begin{align}
  \fdv{f(x, t)}{f(x, t)} & = 1 \implies \delta{f(x, t)} = \int_A^B \odif{x'} \, \fdv{f(x,t)}{f(x',t)} \delta{f(x',t)}
\end{align}
Comparing with this with the definition of \emph{Dirac delta function}:
\dfn{Dirac Delta Function}{
  The \emph{Dirac delta function} $\delta(x)$ is defined as a function that satisfies the following property:
  \begin{align}
    \int \odif{x'} \, \delta(x' - x) \varphi(x') & = \varphi(x), \quad ^{\forall} \varphi(x') \in C^\infty
  \end{align}
}
we have
\thm{Property of Functional Derivative}{
  For a functional $f_i(t)$ or $f(x, t)$, the functional derivative satisfies the following property:
  \begin{align}
    \fdv{f_i(t)}{f_j(t)} = \delta_{ij},  \quad \text{or} \quad
    \fdv{f(x, t)}{f(x', t)} & = \delta(x - x')
    \label{eq:func_deriv_delta}
  \end{align}
}

\subsection{In a n-dimensional space}
In the previous discussions, we have considered the functional $I$ as an integral on 1D space represented by $x$.
Here, we aim to generalize the discussion to $n$-dimensional space, e.g. $\real^n$.
For $\real^n$, let us define a functional $I$ and functional derivative as follows:
\dfn{Functional on $\real^n$}{
  A quantity $I \in \real$ defined on $V \subset \real^n$ is called a \emph{(linear) functional} if it can be expressed as follows:
  \begin{align}
    I[f_i] := \int_{x \in V} \odif[n]{x} \, F(f_i(x)) = \int_{V} \odif[n]{x} \, F(f_i(x)),
    \quad i \in \ntrl
  \end{align}
}
\dfn{Functional Derivative on $\real^n$}{
  The \emph{functional derivative} of a functional $I[f_i]$ with respect to $f_i(x)$ is defined as follows:
  \begin{align}
    \fdv{I[f_j]}{f_i(x)} := \phi_i(x) \iffdef \fdif{I} = \int_V \odif[n]{x'} \, \fdv{I[f_j]}{f_i(x')} \fdif{f_i}(x')
  \end{align}
  where $\phi_i(x)$ is a function of $x$.
}
Then, the variation of the functional $\fdif{I}$ can be expressed as:
\begin{align}
  \fdif{I} = \delta \int_V \odif[n]{x'} \, F(f_j(x'))
   & = \int_V \odif[n]{x'} \, \fdif{F}(f_j(x'))
  = \int_V \odif[n]{x'} \, \sum_j \pdv{F}{f_j(x')} \fdif{f_j}(x')
\end{align}
Now,
\begin{align}
  \delta f_j(x')                                       & = \int_V \odif[n]{x} \, \delta^n(x - x')\delta f_j(x)                         \\
                                                       & = \int_V \odif[n]{x} \,  \sum_i \delta_{ij} \, \delta^n(x - x') \delta f_i(x) \\
  \implies \quad  \frac{\delta f_j(x')}{\delta f_i(x)} & = \delta_{ij} \, \delta^n(x - x')
\end{align}
so,
\begin{align}
  \fdv{I}{f_i(x)} & = \int_V \odif[n]{x'} \, \pdv{F}{f_i (x')} \delta(x - x') = \pdv{F}{f_i(x)}
\end{align}
thus we see that the functional-density relation still holds:
\thm{Functional-density relation}{
  For a quantity $I[F]$ and its density $F[f_i(x)]$, the functional derivative satisfies the following relation:
  \begin{align}
    I = \int_V \odif[n]{x} \, F[f_j(x)] \quad \implies \quad \fdv{I[f_j]}{f_i(x)} = \pdv{F[f_j]}{f_i(x)}
  \end{align}
  \label{thm:func_deriv_relation_n}
}

\cite{eman-functionalDerivative}