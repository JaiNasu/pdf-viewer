\section{Invariance of Euler-Lagrange Equation}
\label{sec:invariance-euler-lagrange}
Let us look at how the Euler-Lagrange Equation transforms under a point transformation of the generalized coordinates.
The Euler-Lagrange Equation in the old coordinates $q_i$ is given by:
\begin{align}
  \pdv{L}{q_i} - \odv{}{t} \pab{\pdv{L}{\dot{q}_i}} & = 0
\end{align}
\mprop{Euler-Lagrange Equation}{
  In the new coordinates $Q_j$, the Euler-Lagrange Equation still holds:
  \begin{align}
    \pdv{L}{Q_j} - \odv{}{t} \pab{\pdv{L}{\dot{Q}_j}} & = 0
  \end{align}
  where $L(Q_j, \dot{Q}_j, t)$ is the Lagrangian written in the new coordinates.
}
Under the point transformation:
\begin{align}
  Q_j & = Q_j(q_1, q_2, \ldots, q_n, t) = Q_j(q_i, t)
\end{align}
the Lagrangian $L$ transforms as follows:
\begin{align}
  \tilde{L} (Q, \dot{Q}, t) & = L(q(Q, t), \dot{q}(Q, \dot{Q}, t), t)
\end{align}
Let us consider how the new coordinates $\dot{Q}_j$ can be written in terms of the old coordinates $\dot{q}_i$:
\begin{align}
  \dot{Q_j} = \odv{Q_j}{t} & = \pdv{Q_j}{t} + \sum_k \pdv{Q_j}{q_k} \dot{q}_k \implies \pdv{\dot{Q_j}}{\dot{q}_i} = \pdv{Q_j}{q_i}
  \label{eq:derivative-derivative-1}
\end{align}
using this,
\begin{align}
  \odv{}{t} \pab{\pdv{\dot{Q}_j}{\dot{q}_i}}
   & = \odv{}{t} \pab{\pdv{Q_j}{q_i}}  = \pdv{Q_j}{t, q_i} + \sum_k \pdv{Q_j}{q_k, q_i}  = \pdv{Q_j}{q_i, t} + \sum_k \pdv{Q_j}{q_i, q_k}
\end{align}
and
\begin{align}
  \pdv{\dot{Q}_j}{q_i} & = \pdv{}{q_i} \pab{\pdv{Q_j}{t} + \sum_k \pdv{Q_j}{q_k}} = \pdv{Q_j}{q_i, t} + \sum_k \pdv{Q_j}{q_i, q_k} = \odv{}{t} \pab{\pdv{\dot{Q}_j}{\dot{q}_i}}
  \label{eq:derivative-derivative-2}
\end{align}
Now, we try to write the Euler-Lagrange Equation in the old coordinates in terms of the new coordinates:
\begin{align}
  \pdv{L(Q_j, \dot{Q}_j, t)}{q_i} & = \sum_j \pdv{L}{Q_j} \pdv{Q_j}{q_i} + \pdv{L}{\dot{Q}_j} \pdv{\dot{Q}_j}{q_i} + \pdv{L}{t} \pdv{t}{q_i}
\end{align}
Now, time $t$ is an independent variable: a parameter that "we" set to describe the system, and crucially does not depend on the generalized coordinates $q_i$ or the generalized velocities $\dot{q}_i$.
Thus, we have $\pdv{t}{q_i} = 0$.
\begin{align}
  \implies \quad \pdv{L(Q_j, \dot{Q}_j, t)}{q_i} & = \sum_j \pdv{L}{Q_j} \pdv{Q_j}{q_i} + \pdv{L}{\dot{Q}_j} \pdv{\dot{Q}_j}{q_i}
  \intertext{and}
  \pdv{L(Q_j, \dot{Q}_j, t)}{\dot{q}_i}          & = \sum_j \pdv{L}{Q_j} \pdv{Q_j}{\dot{q}_i} + \pdv{L}{\dot{Q}_j} \pdv{\dot{Q}_j}{\dot{q}_i}
\end{align}
since $Q_j(q_i, t)$ is not a function of $\dot{q}_i$, we have $\pdv{Q_j}{\dot{q}_i} = 0$.
\begin{align}
  \implies \quad \pdv{L(Q_j, \dot{Q}_j, t)}{\dot{q}_i}      & = \sum_j \pdv{L}{\dot{Q}_j} \pdv{\dot{Q}_j}{\dot{q}_i}                                                                                 \\
  \implies \quad \odv{}{t} \pab{\pdv{\dot{Q}_j}{\dot{q}_i}} & = \sum_j \odv{}{t} \pab{\pdv{L}{\dot{Q}_j}} \pdv{\dot{Q}_j}{\dot{q}_i} + \pdv{L}{\dot{Q}_j} \odv{}{t} \pab{\pdv{\dot{Q}_j}{\dot{q}_i}}
\end{align}
So, if we compare each term of the Euler-Lagrange Equation in the old coordinates,
\begin{align}
  \odv{}{t} \pab{\pdv{L}{\dot{q}_i}} & = \sum_j \odv{}{t} \pab{\pdv{L}{\dot{Q}_j}} \pdv{\dot{Q}_j}{\dot{q}_i} + \pdv{L}{\dot{Q}_j} \odv{}{t} \pab{\pdv{\dot{Q}_j}{\dot{q}_i}} \\
  \pdv{L}{q_i}                       & = \sum_j \phantom{\odv{}{t}} \pdv{L}{Q_j} \hspace{0.7cm} \pdv{Q_j}{q_i} + \pdv{L}{\dot{Q}_j} \phantom{\odv{}{t}} \pdv{\dot{Q}_j}{q_i}
\end{align}
from Eq. \eqref{eq:derivative-derivative-1} and Eq. \eqref{eq:derivative-derivative-2}, we can see that
\begin{align}
  \pdv{L}{q_i} - \odv{}{t} \pab{\pdv{L}{\dot{q}_i}} & = \sum_j \pdv{Q_j}{q_i} \bab{\pdv{L}{Q_j} - \odv{}{t} \pab{\pdv{L}{\dot{Q}_j}}} = 0
  \label{eq:euler-lagrange-invariance}
\end{align}
Now, if we remember that the (i-th) component of the product of a matrix and a vector is given by
\begin{align}
  (A \vec{x})_i = \sum_j A_{ij} x_j
\end{align}
So Eq. \eqref{eq:euler-lagrange-invariance} can be rewritten as a product of a matrix and a vector:
\begin{align}
  \sum_j J_{ij} x_j & = 0 \quad \text{where} \quad J_{ij} = \pdv{Q_j}{q_i}, \quad x_j = \pdv{L}{Q_j} - \odv{}{t} \pab{\pdv{L}{\dot{Q}_j}}
\end{align}
As long as the point transformation is invertible(i.e. changing from 3D Cartesian coordinates to 3D spherical coordinates), the Jacobian matrix $J_{ij}$ is invertible, and thus the only solution to the equation above is $x_j = 0$ for all $j$.
Thus, we have
\begin{align}
  \pdv{L}{Q_j} - \odv{}{t} \pab{\pdv{L}{\dot{Q}_j}} & = 0
\end{align}




\cite{hachiware-analyticalMechanics}