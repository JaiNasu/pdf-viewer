\section{Position and Momentum as Observables}
\subsection{Position Operator}
As an observable, let us consider the position of a particle: $\hat{x}$.
We can set up the position operator $\hat{x}$ in the Hilbert space $\hilbert$ and its eigenstates $\ket{x}$, which satisfy the eigenvalue equation
\begin{align}
  \hat{x} \ket{x} = x \ket{x}.
\end{align}
As we have seen in continous case of Sec \ref{sec:braket-notation}, the inner product $\brac{x}{y}$ is defined as
\begin{align}
  \braket{x}{x^\prime} = \delta(x - x^\prime),
\end{align}
and similarly to the discrete case, for a state $\ket{\psi}$, its probability density $P(x)$ is given by
\begin{align}
  P(x) = \abs{\braket{x}{\psi}}^2 = \abs{\psi(x)}^2,
\end{align}
where we defined the \emph{wavefunction}:
\dfn{wavefunction}{
  The \emph{wavefunction} $\psi(x)$ of a quantum state $\ket{\psi}$ is a function that describes the probability amplitude of finding the particle at position $x$.
  \begin{align}
    \psi(x) := \braket{x}{\psi}
  \end{align}
}

\subsection{Momentum Operator}
We can also consider the momentum of a particle $p$ as an observable.
Really, the same discussion applies to the momentum operator $\hat{p}$.
It is defined in the Hilbert space $\hilbert$ and has eigenstates $\ket{p}$ that satisfy the eigenvalue equation
\begin{align}
  \hat{p} \ket{p} = p \ket{p}.
\end{align}
the completeness relation reads
\begin{align}
  \int \odif{p} \, \ket{p} \bra{p} = \idty \iff \braket{p}{p^\prime} = \delta(p - p^\prime)
\end{align}
and the wavefunction $\psi(p)$ in the momentum space is defined as
\begin{align}
  \psi(p) := \braket{p}{\psi}
\end{align}
\dfn{Wavefunction in Momentum Space}{
  The wavefunction in momentum space $\psi(p)$ is a function that describes the probability amplitude of finding the particle with momentum $p$.
  \begin{align}
    \psi(p) := \braket{p}{\psi}
  \end{align}
  This is often called the \emph{momentum representation} of the wavefunction.
}
\subsection{Canonical Commutation Relation}
We define a operation called the \emph{commutator} of two operators $\hat{A}$ and $\hat{B}$ as follows:
\dfn{Commutator}{
  The commutator $\hat{A} \hat{B}$ of two operators $\hat{A}$ and $\hat{B}$ is defined as
  \begin{align}
    \commt{\hat{A}}{\hat{B}} := \hat{A} \hat{B} - \hat{B} \hat{A}.
  \end{align}
}
\nt{
  If the commutator of two operators is zero, they are said to \emph{commute}.
  If the commutator is non-zero, they are said to \emph{not commute}.
}
\thm{Properties of Commutator}{
  For operators $\hat{A}, \hat{B}, \hat{C} \in \hilbert$ and constants $\alpha, \beta, \gamma \in \cmplx$, the following properties hold:
  \begin{itemize}
    \item (Bi-)Linearity:
          \begin{align}
            \commt{\alpha \hat{A} + \beta \hat{B}}{\hat{C}} & = \alpha \commt{\hat{A}}{\hat{C}} + \beta \commt{\hat{B}}{\hat{C}} \\
            \commt{\hat{A}}{\beta \hat{B} + \gamma \hat{C}} & = \beta \commt{\hat{A}}{\hat{B}} + \gamma \commt{\hat{A}}{\hat{C}}
          \end{align}
    \item Anti-symmetry:
          \begin{align}
            \commt{\hat{A}}{\hat{B}} = -\commt{\hat{B}}{\hat{A}}
          \end{align}
  \end{itemize}
}
Now, remember the \emph{Poisson Bracket} from classical mechanics, which is defined as
\dfn{Poisson Bracket}{
  The Poisson bracket of two functions $f$ and $g$ in phase space is defined as
  \begin{align}
    \pobra{f}{g} = \pdv{f}{q} \pdv{g}{p} - \pdv{f}{p} \pdv{g}{q},
  \end{align}
  where $q_i$ and $p_i$ are the generalized coordinates and momenta, respectively.
}
\thm{Properties of Poisson Bracket}{
  For functions $f, g$, the following properties hold:
  \begin{itemize}
    \item (Bi-)Linearity:
          \begin{align}
            \pobra{\alpha f + \beta g}{h} & = \alpha \pobra{f}{h} + \beta \pobra{g}{h} \\
            \pobra{f}{\alpha g + \beta h} & = \alpha \pobra{f}{g} + \beta \pobra{f}{h}
          \end{align}
    \item Anti-symmetry:
          \begin{align}
            \pobra{f}{g} = -\pobra{g}{f}
          \end{align}
    \item Jacobi Identity:
          \begin{align}
            \pobra{f}{\pobra{g}{h}} + \pobra{g}{\pobra{h}{f}} + \pobra{h}{\pobra{f}{g}} = 0
          \end{align}
    \item Coordinate and Momentum Relation:
          \begin{align}
            \pobra{q}{p} = 1, \quad \pobra{q}{q} = \pobra{p}{p} = 0
          \end{align}
  \end{itemize}
}
These two objects have the same algebraic properties if we impose the canonical commutation relation onto the commutator of the position and momentum operators:
\prcp{Canonical Commutation Relation (1D)}{
  The commutator of position operator $\hat{x}$ and momentum operator $\hat{p}$ is defined to be
  \begin{align}
    \commt{\hat{x}}{\hat{p}} = i \hbar.
  \end{align}
}
\thm{Commutator as a derivative}{
  Consider a function $f(x)$ that can be Taylor expanded:
  \begin{align}
    f(x) & = \sum_{n=0}^{\infty} a_n x^n
  \end{align}
  where $a_n = \frac{1}{n!} \pdv{f}{x}(x_0)$ and $x$.
  Then, the commutator of the position operator $\hat{x}$ and the momentum operator $\hat{p}$ can be interpreted as a derivative:
  \begin{align}
    \commt{f(\hat{x})}{{\hat{p}}} & = i \hbar \pdv{f}{x}(\hat{x})
  \end{align}
  where $f(\hat{x})$ refers to the substitution of $\hat{x}$ into the function $f(x)$.
  \label{thm:commutator-as-derivative}
}
\pf{a}{
  We have to prove that this holds for $f(x) = x^n$, from which our goal can be proven easily.
  \begin{align}
    \commt{\hat{x}^n}{\hat{p}} & = \hat{x}^{n-1} \hat{x} \hat{p} - \hat{p} \hat{x}^{n-1} \hat{x}                                                                               \\
                               & = \hat{x}^{n-1} \commt{\hat{x}}{\hat{p}} + \hat{x}^{n-1} \hat{p} \hat{x} - \hat{p} \hat{x}^{n-1} \hat{x}                                      \\
                               & = \hat{x}^{n-1} \commt{\hat{x}}{\hat{p}} + \commt{\hat{x}^{n-1}}{\hat{p}} \hat{x}                                                             \\
                               & = i \hbar \hat{x}^{n-1} + \commt{\hat{x}^{n-1}}{\hat{p}} \hat{x}                                                                              \\
                               & = i \hbar \hat{x}^{n-1} + \pab{i \hbar \hat{x}^{n-2} + \commt{\hat{x}^{n-2}}{\hat{p}} \hat{x}} \hat{x}                                        \\
                               & = 2 i \hbar \hat{x}^{n-1} + \commt{\hat{x}^{n-2}}{\hat{p}} \hat{x}                                                                            \\
                               & = n i \hbar \hat{x}^{n-1} + \underbrace{\commt{\hat{x}^0}{\hat{p}}}_{= 0} \hat{x} = n i \hbar \hat{x}^{n-1} = i \hbar \pdv{(x^n)}{x}(\hat{x})
  \end{align}
  thus, if the function $f$ is given by a Taylor series, we can write
  \begin{align}
    \commt{\hat{x}^n}{\hat{p}} = i \hbar \pdv{f}{x}(\hat{x})
  \end{align}
}


\subsection{Heisenberg's Uncertainty Principle}
\dfn{Variance of an Observable}{
  The variance of an observable $\hat{A}$ in a quantum state $\ket{\psi}$ is defined as
  \begin{align}
    \Delta A^2 := \braketop{\psi}{\hat{A}^2}{\psi} - \pab{\braketop{\psi}{\hat{A}}{\psi}}^2.
  \end{align}
  It measures the spread of the observable's values around its expectation value.
}
\thm{Robertson Inequality}{
  The Robertson inequality states that for any two observables $\hat{A}$ and $\hat{B}$, the variance of $\hat{A}$ and $\hat{B}$ satisfies
  \begin{align}
    \Delta A^2 \Delta B^2 \geq \frac{1}{4} \abs{\commt{\hat{A}}{\hat{B}}}^2
  \end{align}
}
\cor{Heisenberg's Uncertainty Principle}{
  The Heisenberg uncertainty principle is a special case of the Robertson inequality, which states that for position $\hat{x}$ and momentum $\hat{p}$,
  \begin{align}
    \Delta x \Delta p \geq \frac{\hbar}{2}.
  \end{align}
}

What this tells us is not a practical limit of measurements, but a fundamental property of quantum systems, as a result of the non-commutativity of the position and momentum operators.

This also means that if the commutator of two observables $\hat{A}$ and $\hat{B}$ is zero, they can be measured with any precision at the same time.
\dfn{Simultaneous Observable}{
  Two observables $\hat{A}$ and $\hat{B}$ are said to be \emph{simultaneous observables}, if they commute:
  \begin{align}
    \commt{\hat{A}}{\hat{B}} = 0.
  \end{align}
}



\subsection{Derivative on Vectors}
It is a bit arbitrary, but let us consider the following operator:
\begin{align}
  \hat{U}_x (a) & = \sum_{n=0}^{\infty} \frac{1}{n!} \pab{- i \frac{\hat{p}}{\hbar} a}^n
\end{align}
This is indeed an operator (guaranteed by the completeness of Hilbert space).
As a short-hand notation for the infinite sum, we can write it as an exponential:
\begin{align}
  \hat{U}_x (a) & = e^{- i \frac{\hat{p}}{\hbar} a}.
\end{align}
Remember that momentum must be an Hermitian operator ($\hat{p}^\dagger = \hat{p}$), so we  see that by taking the the Hermitian conjugate,
\begin{align}
  \hat{S}^\dagger_p (a): = \pab{\hat{U}_x (a)}^\dagger & = \sum_{n=0}^{\infty} \frac{1}{n!} \pab{i \frac{\hat{p}}{\hbar} a}^n = \hat{S}_{p} (- a)
\end{align}
and as you might guess from the exponential form,
\begin{align}
  \hat{U}_x(a) \hat{U}_x (-a)
   & = \pab{1 - i \frac{\hat{p}}{h} a + \frac{1}{2!} \pab{- i \frac{\hat{p}}{\hbar} a}^2 + \ldots} \pab{1 + i \frac{\hat{p}}{\hbar} a + \frac{1}{2!} \pab{i \frac{\hat{p}}{\hbar} a}^2 + \ldots} \\
   & = 1 \pab{- i \frac{\hat{p}}{\hbar} a + i \frac{\hat{p}}{\hbar} a + \frac{1}{2!} \pab{- i \frac{\hat{p}}{\hbar} a}^2 + \frac{1}{2!} \pab{i \frac{\hat{p}}{\hbar} a}^2 + \ldots} = 1
\end{align}
thus we see that
\begin{align}
  \hat{U}_x(a) \hat{U}_x (-a) = \idty \iff \hat{U}_x (a) = \pab{\hat{U}_x(a)}^\dagger = \hat{U}_x (-a).
\end{align}
which means that $\hat{U}_x(a)$ is a \emph{unitary operator}.
Note that unitary operators do not change the norm of a vector, i.e. $\norm{\hat{U}_x(a) \ket{\psi}} = \norm{\ket{\psi}}$.
Now, notice that this operator can be differentiated by the parameter $a$:
\begin{align}
  \odv{}{a} \pab{\hat{U}_x (a)} & = -i \frac{\hat{p}}{\hbar} \hat{U}_x (a), \odv{}{a} \pab{\hat{U}_x (a)}^\dagger = i \frac{\hat{p}}{\hbar} \hat{U}_x (a)^\dagger
\end{align}
\nt{
  The momentum operator $\hat{p}$ and $\hat{U}_x(a)$ commute:
  since $\commt{\hat{p}^n}{\hat{p}} = 0$ and $\hat{U}_x (a)$ is essentially a linear combination of $\hat{p}^n$. Thus,
  \begin{align}
    \commt{\hat{U}_x(a)}{\hat{p}} = 0 \iff \hat{U}_x(a) \hat{p} = \hat{p} \hat{U}_x(a)
  \end{align}
  meaning that we can change the order of the said operators.
}
thus, by considering the following quantity:
\begin{align}
  \odv{}{a} \pab{\hat{S}^\dagger_p (a) \hat{x} \pab{\hat{U}_x(a)}}
   & = \odv{}{a} \pab{\hat{S}^\dagger_p (a)} \hat{x} \hat{U}_x (a) + \hat{U}_x (-a) \hat{x} \odv{}{a} \pab{\hat{U}_x (a)}                             \\
   & = \hat{S}^\dagger_p (a) \pab{i \frac{\hat{p}}{\hbar}} \hat{x} \hat{U}_x (a) - \hat{U}_x (-a) \hat{x} \pab{i \frac{\hat{p}}{\hbar}} \hat{U}_x (a) \\
   & = \hat{S}^\dagger_p (a) \commt{i \frac{\hat{p}}{\hbar}}{\hat{x}} \hat{U}_x (a)                                                                   \\
   & = \frac{1}{i \hbar} \hat{S}^\dagger_p (a) \underbrace{\commt{\hat{x}}{\hat{p}}}_{= i \hbar} \hat{U}_x (a)                                        \\
   & = \hat{S}^\dagger_p (a) \hat{U}_x (a) = \idty                                                                                                    \\
  \implies \quad \hat{S}^\dagger_p (a) \hat{x} \pab{\hat{U}_x(a)}
   & = a \idty + \hat{C}
\end{align}
where $\hat{C}$ is a constant operator that does not depend on $a$.
If we substitute $a = 0$, we see that
\begin{align}
  \hat{U}_x (0) = \hat{S}^\dagger_p (0) = \idty \implies \hat{C} = \hat{x}
\end{align}
Thus, we have the following relation:
\begin{align}
  \hat{S}^\dagger_p (a) \hat{x} \, \hat{U}_x(a) & = a \idty + \hat{x}
  \implies \quad \hat{x} \, \hat{U}_x(a)        & = a \hat{U}_x(a) + \hat{U}_x(a) \hat{x} \\
  \implies \quad \commt{\hat{s}}{\hat{x}}       & = a \hat{U}_x(a)
\end{align}
thus by applying $\hat{x} \hat{U}_x(a)$ to $\ket{x}$,
\begin{align}
  \hat{x} \hat{U}_x(a) \ket{x}
   & = (\commt{\hat{x}}{\hat{U}_x(a)} + \hat{U}_x(a) \hat{x}) \ket{x} \\
   & = a \hat{U}_x(a) \ket{x} + \hat{U}_x(a) \hat{x} \ket{x}          \\
   & = (x + a) \hat{U}_x(a)\ket{x}
\end{align}
thus the vector $\hat{U}_x(a) \ket{x}$ behaves like $\ket{x + a}$(whose eigenvalue for $\hat{x}$ is $x + a$).
and so we call $\hat{U}_x(a)$ the \emph{shift operator}:
\dfn{Shift Operator}{
  The shift operator $\hat{U}_x(a)$ is defined as
  \begin{align}
    \hat{U}_x(a) = e^{- i \frac{\hat{p}}{\hbar} a},
  \end{align}
  and it shifts the position eigenstate $\ket{x}$ by $a$:
  \begin{align}
    \hat{U}_x(a) \ket{x} = \ket{x + a}.
  \end{align}
  It is a unitary operator that commutes with the momentum operator $\hat{p}$.
}

Now, notice that when $a$ is small, let's say $a = \epsilon$, we can expand the exponential:
\begin{align}
  \hat{U}_x (\epsilon) & = 1 - i \frac{\hat{p}}{\hbar} \epsilon + \frac{1}{2!} \pab{- i \frac{\hat{p}}{\hbar} \epsilon}^2 + \ldots
\end{align}
if we take $\epsilon$ small enough, the $\epsilon^2, \epsilon^3, \ldots$ terms will be much smaller than the $\epsilon$ term, so:
\begin{align}
  \hat{U}_x (\epsilon) & \approx 1 - i \frac{\hat{p}}{\hbar} \epsilon
\end{align}
thus by applying this to a vector $\ket{x}$, we have
\begin{align}
  \hat{U}_x (\epsilon) \ket{x}                                 & \approx \pab{1 - i \frac{\hat{p}}{\hbar} \epsilon} \ket{x} = \ket{x + \epsilon} \\
  \implies \quad \frac{\ket{x + \epsilon} - \ket{x}}{\epsilon} & \approx - i \frac{\hat{p}}{\hbar} \ket{x}
\end{align}
the LHS is the \emph{derivative} of the vector $\ket{x}$ with respect to the parameter $\epsilon$:
\dfn{Derivative of a Vector}{
  The derivative of a continous eigenvector $\ket{x}$ with respect to $x$ is defined as
  \begin{align}
    \odv{}{x} \ket{x} := \lim_{\epsilon \to 0} \frac{\ket{x + \epsilon} - \ket{x}}{\epsilon}
  \end{align}
}
Thus, we can write the above relation as
\begin{align}
  \pdv{}{x} \ket{x} & = - i \frac{\hat{p}}{\hbar} \ket{x} \iff \hat{p} \ket{x} = i \hbar \odv{}{x} \ket{x}
\end{align}
taking the Hermitian conjugate,
\begin{align}
  \bra{x} \hat{p} & = - i \hbar \pdv{}{x} \bra{x}
\end{align}
given a state $\ket{\psi}$, this can be treated as the derivative on the wavefunction $\psi(x) = \braket{x}{\psi}$:
\begin{align}
  \braketop{x}{\hat{p}}{\psi} & = - i \hbar \pdv{}{x} \braket{x}{\psi} = - i \hbar \pdv{}{x} \psi(x).
\end{align}
This is called the position representation of the momentum operator:
\begin{align}
  \hat{p} \ket{x} = i \hbar \pdv{}{x} \ket{x}
\end{align}
because in the momentum representation $\hat{p}$ acts as a multiplication operator:
\begin{align}
  \hat{p} \ket{p} & = p \ket{p} \implies \hat{p} = p.
\end{align}
and by the duality of commutator, $\hat{x}$ acts as a derivative operator in the momentum representation:
\begin{align}
  \hat{x} = i \hbar \pdv{}{p}.
\end{align}

\subsection{3D Case}
In the 3D case, we can define the position operator $\hat{\vec{x}}$ and momentum operator $\hat{\vec{p}}$ as vectors:
\begin{align}
  \hat{\vec{x}} & = \pab{\hat{x}_1, \hat{x}_2, \hat{x}_3}, \quad \hat{\vec{p}} = \pab{\hat{p}_1, \hat{p}_2, \hat{p}_3}
\end{align}
which has eigenvectors corresponding to each position and momentum vectors:
\begin{align}
  \hat{\vec{x}} \ket{\vec{x}} & = \vec{x} \ket{\vec{x}}, \quad \hat{\vec{p}} \ket{\vec{p}} = \vec{p} \ket{\vec{p}}
\end{align}
The canonical commutation relation in 3D is given by
\begin{align}
  \commt{\hat{x}_i}{\hat{p}_j} = i \hbar \delta_{ij}, \quad i, j = 1, 2, 3.
\end{align}
and the momentum operator in the position representation is given by
\begin{align}
  \hat{\vec{p}} = - i \hbar \nabla, \quad \text{where } \nabla = \begin{pmatrix}
                                                                   \pdif{1} \\
                                                                   \pdif{2} \\
                                                                   \pdif{3}
                                                                 \end{pmatrix}
\end{align}
Essentially, the same discussion applies to the 3D case as well.


Now, in 3D, we can define another important quantity in quantum mechanics: the \emph{orbital angular momentum} operator, which is defined as follows:
\dfn{Orbital Angular Momentum Operator}{
  The orbital angular momentum operator $\hat{\vec{L}}$ is defined as
  \begin{align}
    \hat{\vec{L}} = \hat{\vec{x}} \times \hat{\vec{p}} = \begin{pmatrix}
                                                           \hat{L}_x \\
                                                           \hat{L}_y \\
                                                           \hat{L}_z
                                                         \end{pmatrix} = \begin{pmatrix}
                                                                           \hat{x}_2 \hat{p}_3 - \hat{x}_3 \hat{p}_2 \\
                                                                           \hat{x}_3 \hat{p}_1 - \hat{x}_1 \hat{p}_3 \\
                                                                           \hat{x}_1 \hat{p}_2 - \hat{x}_2 \hat{p}_1
                                                                         \end{pmatrix}.
  \end{align}
  or equivalently,
  \begin{align}
    \hat{L}_i = \textcolor{gray!50}{\sum_{j, k = 1}^{3}} \epsilon_{ijk} \hat{x}_j \hat{p}_k, \quad i, j, k = 1, 2, 3.
  \end{align}
}
and the total orbital angular momentum operator $\hat{\vec{L}}^2$:
\dfn{Total Orbital Angular Momentum Operator}{
  The total orbital angular momentum operator $\hat{\vec{L}}^2$ is defined as
  \begin{align}
    \hat{\vec{L}}^2 = \hat{L}_x^2 + \hat{L}_y^2 + \hat{L}_z^2.
  \end{align}
}
\thm{Properties of Angular Momentum Operators}{
  \begin{align}
    \commt{\hat{L}_i}{\hat{L}_j}       & = i \hbar \epsilon_{ijk} \hat{L}_k, \quad i, j, k = 1, 2, 3 \\
    \commt{\hat{\vec{L^2}}}{\hat{L}_i} & = 0
  \end{align}
}
These properties imply two important properties:
\begin{itemize}
  \item Any two(or all three) components of the angular momentum operator $\hat{L}_i$ do not commute, which means that they cannot be measured simultaneously with arbitrary precision.
  \item However, the total angular momentum operator $\hat{\vec{L^2}}$ commutes with any component of the angular momentum operator $\hat{L}_i$, which means that we can measure the total angular momentum and one component of it simultaneously with arbitrary precision.
\end{itemize}

\section{Heisenberg Equation and Schrödinger Equation}
Now, we have postulated the canonical commmutation from the Poisson brackets in classical mechanics:
\begin{align}
  \pobra{x}{p} = 1 \longrightarrow \commt{\hat{x}}{\hat{p}} = i \hbar
\end{align}
and so we postulate that in general, there exists the following correspondence:
\begin{align}
  \pobra{A}{B} = C \longrightarrow \commt{\hat{A}}{\hat{B}} = i \hbar \hat{C}
\end{align}
One example is:
\begin{align}
  \pobra{A}{H} = \odv{A}{t} \longrightarrow \commt{\hat{A}(t)}{\hat{H}} = i \hbar \odv{\hat{A}}{t}
\end{align}
This is called the \emph{Heisenberg Equation}, and at the same time,
it is insinuated that what really evolves through time is the operator $\hat{A}$ (the outcomes), and the state (or probability distribution of the system) is fixed:
\prcp{Heisenberg Picture and Heisenberg Equation}{
  The point of view that the operators evolve through time, while the states are fixed is called the \emph{Heisenberg picture}.
  The time evolution of an operator $\hat{A}(t)$ in the Heisenberg picture is given by the \emph{Heisenberg Equation}:
  \begin{align}
    \commt{\hat{A}(t)}{\hat{H}} = i \hbar \odv{\hat{A}(t)}{t}
  \end{align}
  note: we assume that at $t = 0$, $\hat{A}(0) = \hat{A}$.
}
in contrast, we have another point of view:
\prcp{Schrödinger Picture}{
  The point of view that the states evolve through time, while the operators are fixed is called the \emph{Schrödinger picture}.
  We denote the time evolved states as $\ket{\psi(t)}$,
}
If we remember the Theorem \ref{thm:commutator-as-derivative}:
\begin{align}
  \commt{f(\hat{x})}{\hat{p}} = i \hbar \pdv{f}{x}(\hat{x}), \quad
  \hat{p}\ket{x} = i \hbar \pdv{}{x} \ket{x}.
\end{align}
and the shift operator was defined as
\begin{align}
  \hat{U}_x(a) = e^{- i \frac{\hat{p}}{\hbar} a}
\end{align}
we postulate then that by changing $\hat{p} \to \hat{H}$ and $\pdv{}{x} \to \odv{}{t}$, we can define the time evolution operator:
\dfn{Time Evolution Operator}{
  The time evolution operator $\hat{U}(t)$ is defined as
  \begin{align}
    \hat{U}(t) = e^{- i \frac{\hat{H}}{\hbar} t},
  \end{align}
  where $\hat{H}$ is the Hamiltonian operator of the system.
}
and hence the time evolution of a state $\ket{\psi(0)}$ is given by
\begin{align}
  \ket{\psi(t)} = \hat{U}(t) \ket{\psi(0)}
\end{align}
thus by taking the time derivative on both sides, we obtain the \emph{Schrödinger Equation}:
\prcp{Schrödinger Equation}{
  The \emph{Schrödinger Equation} is given by
  \begin{align}
    i \hbar \odv{}{t} \ket{\psi(t)} = \hat{H} \ket{\psi(t)},
  \end{align}
  where $\hat{H}$ is the Hamiltonian operator of the system.
}
Note that the Hamiltonian operator $\hat{H}$ corresponds to the Hamiltonian in the classical mechanics.
Also, if we apply $\bra{x}$, we can obtain the Schrödinger Equation we are familiar with:
\begin{align}
  i \hbar \pdv{}{t} \psi(x, t) = \hat{H} \psi(x, t),
\end{align}
where $\psi(x, t) = \braket{x}{\psi(t)}$ is the wavefunction in the position representation.
In the case where the Hamiltonian is time-independent, since the Hamiltonian is the energy of the system in the classcal case, its eigenvalues should give the energy of the system:
\dfn{Time-Independent Schrödinger Equation}{
  The time-independent Schrödinger equation is given by
  \begin{align}
    \hat{H} \psi(x) = E \psi(x),
  \end{align}
  where $E$ is the energy eigenvalue of the system.
}
Coming back to the Heisenberg picture, we expect that the expectation value of $\hat{A}(t)$ and $\hat{A}$ in state $\ket{\psi(t)}$ are the same:
\begin{align}
  \braketop{\psi(0)}{\hat{A}(t)}{\psi(0)} & = \braketop{\psi(t)}{\hat{A}}{\psi(t)}                               \\
                                          & = \braketop{\psi(0)}{\hat{U}^\dagger(t) \hat{A} \hat{U}(t)}{\psi(0)}
\end{align}
Thus in the Heisenberg picture, the time evolution of operator $\hat{A}(t)$ is given by
\begin{align}
  \hat{A}(t) & = \hat{U}^\dagger(t) \hat{A} \hat{U}(t).
\end{align}