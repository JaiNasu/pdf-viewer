\newpage
\section{Schrödinger Equation - Original Formulation}
\subsection{Historical Background}
In 1905, Albert Einstein found the photoelectric effect, suggesting that light has a particle like property, such as discrete energy
\cite{1905-photoelectric}.
Following this work, Arthur Compton in 1923 discovered the Compton effect, which is the scattering of X-rays by electrons, further supporting the particle nature of light and confirming the discrete momentum of photons
\cite{1923-compton}.
Specifically, energy and momentum are each related to the frequency and wavelength of light, respectively, as follows:
\mprop{Planck-Einstein relation}{
  For a photon with frequency $\nu$ and wave length $\lambda$, the energy $E$ and momentum $p$ are given by
  \begin{align}
    E = h \nu, \quad p = \frac{h}{\lambda} = \frac{h \nu}{c} = \frac{E}{c}, \label{eq:photon-energy-momentum}
  \end{align}
}
In 1925, Louis de Brogile posulated that other particles (or any matter) also have a wave-like property, and the energy-frequency/ momentum-wavelength relation is given by the Planck-Einstein relation Eq. \eqref{eq:photon-energy-momentum} \cite{1925-deBroglie}.

Then in 1913, Niels Bohr proposed a model of the hydrogen atom, which describes the electron as a particle orbiting the nucleus in discrete energy levels \cite{1913-bohr}.
The implication of this model is that electrons have a fixed, discrete(quantized) angular momentum $\vec{L} := \vec{r} \times \vec{p}$.
For a particle orbiting in a circular orbit, the angular momentum is given by
\begin{align}
  \vec{L} & = r p = \frac{h r}{\lambda}
\end{align}
For the electron wave to be continous around the orbit of radius $r$, the wavelength must be an integer multiple of the circumference of the orbit:
\begin{align}
  \lambda & = \frac{2 \pi r}{n}, \quad n = 1, 2, 3, \ldots \implies \vec{L} = n \frac{h}{2 \pi} := n \hbar
\end{align}

\subsection{Schrödinger Equation}
Now, a classical wave obeys a wave equation:
\begin{align}
  \frac{1}{v^2} \pdv[2]{\psi}{t} - \nabla^2 \psi & = 0
\end{align}
but does the matter wave also obey wave equation?

Well, not a classical wave equation, but a quantum wave should follow a quantum wave equation, which must satisfy a few conditions that match with the physics as we know it:
\begin{itemize}
  \item Planck-Einstein relation: $E = \hbar \omega$, $\vec{p} = \hbar \vec{k}$
  \item Energy equation: $E = T + V$
  \item For $V = V_0$, the solution must be a plane wave: $\psi(\vec{x}, t) = A e^{i (\vec{k} \cdot \vec{x} - \omega t)}$(i.e. constant momentum and constant energy).
\end{itemize}
The energy equation is given by
\begin{align}
  E & = T + V = \frac{p^2}{2m} + V_0 \iff \hbar \omega = \frac{\hbar^2 \vec{k}^{\,2}}{2m} + V
\end{align}
If we multiply both sides by $\psi(\vec{x}, t)$, we get
\begin{align}
  \hbar \omega \psi(\vec{x}, t) & = \frac{\hbar^2}{2m} \vec{k}^{\,2} \psi(\vec{x}, t) + V \psi(\vec{x}, t)
\end{align}
For a plane wave solution,
\begin{alignat}{2}
  \pdv{}{t} \psi(\vec{x}, t)                     & = - i \omega \psi(\vec{x}, t)
                                                 & \quad \implies
  i \hbar \pdv{}{t} \psi(\vec{x}, t)             & = \hbar \omega \psi(\vec{x}, t)                      \\
  \nabla^2 \psi(\vec{x}, t)                      & = - \vec{k}^{\,2} \psi(\vec{x}, t)
                                                 & \quad \implies
  - \frac{\hbar^2}{2m} \nabla^2 \psi(\vec{x}, t) & =  \frac{\hbar^2 \vec{k}^{\,2}}{2m} \psi(\vec{x}, t)
\end{alignat}
Thus, we can write the wave equation as
\begin{align}
  i \hbar \pdv{}{t} \psi(\vec{x}, t) & = - \frac{\hbar^2}{2m} \nabla^2 \psi(\vec{x}, t) + V_0 \psi(\vec{x}, t)
\end{align}
And we postulate that for any potential $V(\vec{x})$, the wave equation is given by
\prcp{Schrödinger Equation}{
  The Schrödinger equation is given by
  \begin{align}
    i \hbar \pdv{}{t} \psi(\vec{x}, t) & = - \frac{\hbar^2}{2m} \nabla^2 \psi(\vec{x}, t) + V(\vec{x}) \psi(\vec{x}, t)
  \end{align}
  where $\psi(\vec{x}, t)$ is the wave function, $V(\vec{x})$ is the potential, and $m$ is the mass of the particle.
}
