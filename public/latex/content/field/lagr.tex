\section{Lagrange Formalism}
\subsection{Potential and Momentum Density}
Now, notice that the force on the infinitisimal section is given by
\begin{align}
  \dot{p} = \mu \pdif[2]{t} \psi \Delta x
\end{align}
since the extension of the string in $\Delta x /2$ is given by
\begin{align}
  \Delta l - \frac{\Delta x}{2} & = \frac{\Delta x}{2} \sqrt{1 + \bab{\pdif{x} \psi\pab{\xi_0 -\frac{\Delta x}{2}}}^2} - \frac{\Delta x}{2} \\
                                & \approx \frac{\Delta x}{2} \pdif{x} \psi\pab{\xi_0 - \frac{\Delta x}{2}}
\end{align}
The force on the infinitisimal section due to the extension is then given by
\begin{align}
  F                & = - k \pab{\Delta l - \frac{\Delta x}{2}}
  = - k \frac{\Delta x}{2} \pdif{x} \psi\pab{\xi_0 - \frac{\Delta x}{2}}
  = - T \pdif{x} \psi\pab{\xi_0 - \frac{\Delta x}{2}}                    \\
  \implies \quad V & = \frac{k}{2} \pab{\Delta l - \frac{\Delta x}{2}}^2
  = \frac{k}{2} \pab{\frac{\Delta x}{2} \pdif{x} \psi\pab{\xi_0 - \frac{\Delta x}{2}}}^2
  = \frac{T}{2} \frac{\Delta x}{2} \pab{\pdif{x} \psi\pab{\xi_0 - \frac{\Delta x}{2}}}^2
\end{align}
We should define these quantity as the potential energy density and the momentum density:
\dfn{Momentum and Potential Energy Density}{
  The \emph{momentum density} and \emph{potential energy density} is defined as the momentum and potential energy per unit length of the string:
  \begin{align}
    \pi & = \mu \pdif{t} \psi, \quad \tilde{V} = \frac{T}{2} (\pdif{x} \psi)^2
  \end{align}
}
Now, since we have defined our potential energy (density) and momentum (density), we can write the Lagrangian:
\begin{align}
  L & = T - V                                                                                    \\
    & = \int \odif{x} \, \tilde{T} - \int \odif{x} \, \tilde{V}                                  \\
    & = \int \odif{x} \, \frac{\mu}{2} \pab{\pdif{t} \psi}^2 - \frac{T}{2} \pab{\pdif{x} \psi}^2
\end{align}
thus we should define the Lagrangian density $\lagr$ as:
\dfn{Lagrangian Density for 1D String}{
  The \emph{Lagrangian density} is defined as the Lagrangian per unit length of the string:
  \begin{align}
    \lagr[\pdif{x} \psi, \pdif{t} \psi] & = \frac{\mu}{2} \pab{\pdif{t} \psi}^2 - \frac{T}{2} \pab{\pdif{x} \psi}^2
  \end{align}
}
Then the Euler-Lagrange equation for the Lagrangian density $\lagr$ is given by:
\begin{alignat}{2}
   &          & \pdif{t} \pab{\pdv{\lagr}{(\pdif{t} \psi)}} + \pdif{x} \pab{\pdv{\lagr}{(\pdif{x} \psi)}} & = \pdv{\lagr}{\psi}              \\
   & \implies & \quad \pdif{t} \pab{\mu \pdif{t} \psi} + \pdif{x} \pab{- T \pdif{x} \psi}                 & = 0                              \\
   & \implies & \quad \mu \pdif[2]{t} \psi - T \pdif[2]{x} \psi                                           & = 0                              \\
   & \implies & \quad \pdif[2]{x} \psi                                                                    & = \frac{\mu}{T} \pdif[2]{t} \psi
\end{alignat}
which is the wave equation in 1D.
Notice that our mechanical variable is now the field $\psi (x, t)$ and its derivatives, rather than the position of a particle or the velocity of a particle.

\subsection{Generalization to General Field}
Now, we would like to apply Lagrange formalism to a general field $\psi(\vec{x}, t)$.
We can define the Lagrangian density for a general field $\psi(\vec{x}, t)$ as:
\dfn{Lagrangian Density of a Field}{
  The \emph{Lagrangian density} of a field $\psi(\vec{x}, t)$ is defined such that the Lagrangian is given by the integral of the Lagrangian density over the whole space:
  \begin{align}
    L[\psi, \pdif{t} \psi, \pdif{i} \psi] & = \int \odif{\vec{x}} \, \lagr(\psi, \pdif{\mu} \psi)
  \end{align}
  where for $\mu = 0, 1, 2, 3$, $x^{0,1,2,3} = (t, \vec{x})$
  \begin{align}
    \pdif{\mu} \psi & := \pdv{\psi}{x^\mu} = \pab{\pdv{\psi}{t}, \pdv{\psi}{x^1}, \pdv{\psi}{x^2}, \pdv{\psi}{x^3}}
  \end{align}
}
\dfn{Action of a Field}{
  The \emph{action} of a field $\psi(\vec{x}, t)$ is defined as the integral of the Lagrangian over time:
  \begin{align}
    S[\psi] & = \int \odif{t} \, L[\psi, \pdif{t} \psi, \pdif{i} \psi] = \int \odif{t} \int \odif[3]{\vec{x}} \, \lagr(\psi, \pdif{\mu} \psi)
  \end{align}
  where $\lagr$ is the Lagrangian density of the field $\psi(\vec{x}, t)$.
}
and the stationary point of the action is given by:
\begin{align}
  \fdif{S}[\lagr] = 0 \implies \int \odif{t} \int \odif[3]{\vec{x}} \, \pdv{\lagr}{\psi} \delta \psi + \pdv{\lagr}{(\pdif{\mu} \psi)} \delta (\pdif{\mu} \psi) = 0
\end{align}
where $\delta \psi = 0$ at the boundary of the integration region.
The second term can be integrated by parts:
\begin{align}
  \int \odif{t} \int \odif[3]{\vec{x}} \, \pdv{\lagr}{(\pdif{\mu} \psi)} \delta(\pdif{\mu} \psi)
   & = \int \odif{t} \int \odif[3]{\vec{x}} \, \pdv{\lagr}{(\pdif{\mu} \psi)} \pdif{\mu} (\delta \psi)        \\
   & = \int \odif{t} \int \odif[3]{\vec{x}} \, \pdif{\mu} \pab{\pdv{\lagr}{(\pdif{\mu} \psi)} \delta \psi}
  - \int \odif{t} \int \odif[3]{\vec{x}} \, \pdif{\mu} \pab{\pdv{\lagr}{(\pdif{\mu} \psi)}} \delta \psi
  \intertext{Since $\delta \psi = 0$ at the boundary of the integration region, the first term vanishes:}
   & = -  \int \odif{t} \int \odif[3]{\vec{x}} \, \pdif{\mu} \pab{\pdv{\lagr}{(\pdif{\mu} \psi)}} \delta \psi
\end{align}
Thus, the stationary point of the action is given by:
\begin{align}
  \int \odif{t} \int \odif[3]{\vec{x}} \, \pab{\pdv{\lagr}{\psi} - \pdif{\mu} \pab{\pdv{\lagr}{(\pdif{\mu} \psi)}}} \delta \psi = 0
\end{align}
Since time and space are arbitrary, the integrand must be zero:
\begin{align}
  \pdv{\lagr}{\psi} - \pdif{\mu} \pab{\pdv{\lagr}{(\pdif{\mu} \psi)}} & = 0
  \iff \pdif{t}\pab{\pdv{\lagr}{(\pdif{t} \psi)}} + \sum_i \pdif{i}\pab{\pdv{\lagr}{(\pdif{i} \psi)}} = \pdv{\lagr}{\psi}
\end{align}
This is the \emph{Euler-Lagrange equation} for a field $\psi(\vec{x}, t)$:
\thm{Euler-Lagrange Equation for a Field}{
  The \emph{Euler-Lagrange equation} for a field $\psi(\vec{x}, t)$ is given by:
  \begin{align}
    \pdv{\lagr}{\psi} - \pdif{\mu} \pab{\pdv{\lagr}{(\pdif{\mu} \psi)}} & = 0
  \end{align}
  where $\lagr$ is the Lagrangian density of the field $\psi(\vec{x}, t)$.
}

Additionally, we define the \emph{canonical conjugate field} $\pi (\vec{x}, t)$ as:
\dfn{Canonical Conjugate Field}{
  The \emph{canonical conjugate field} or \emph{canonical momentum density} $\pi(\vec{x}, t)$ is defined as the derivative of the Lagrangian density with respect to the time derivative of the field $\psi$:
  \begin{align}
    \pi(\vec{x}, t) & := \frac{\partial \lagr}{\partial (\pdif{t} \psi(\vec{x}, t))}
  \end{align}
}

