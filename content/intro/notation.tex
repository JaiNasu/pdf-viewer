\section{General Notations}
\label{sec:notations}
\subsection{Vectors and Functions}
A vector (in a non-relativistic context) is denoted by an arrow:
\begin{align}
  \vec{x}, \vec{r}, \vec{p}, \vec{F}, \vec{v}, \vec{a}, \ldots
\end{align}
If a function depends on multiple variables (e.g., $\{q_i\} = \{q_1, q_2, q_3, \ldots\}$), we may denote it in a short-handed way:
\begin{align}
  f(q_1, q_2, q_3, \ldots) \iff f(q_i)
\end{align}
A partial derivative by the variable $x$ is denoted by:
\begin{align}
  \pdv{f}{x} = \pdif{x} f,
\end{align}
For a general coordinate system $(x_1, x_2, x_3, \ldots)$, we denote the partial derivative by:
\begin{align}
  \pdv{f}{x_i} = \pdif{i} f
\end{align}

\section{Relativity Notations}
\subsection{Vectors(in Relativity)}
For 4-vectors and tensors in relativity, sans-serif font is used:
\begin{align}
  \rel{x}, \rel{p}^\mu, \rel{g}_{\mu \nu}
\end{align}
Each component of a 4-vector is denoted by a superscript or subscript:
\begin{align}
  \rel{x}^\mu = x^0, x^1, x^2 \text{ or } x^3
\end{align}
and without an index, it represents the entire vector:
\begin{align}
  \rel{x} = (x^0, x^1, x^2, x^3) = (ct, \vec{x})
\end{align}


\subsection{Einstein's Summation Convention}
If an index such as $i, j, k, \mu, \nu$, etc., appears twice in a single term, it implies summation over that index. For example:
\eg{Examples}{
  \begin{align}
    \vspace*{3em}
    x_i y_i = \vec{x} \cdot \vec{y}           & = \sum_{i = 1}^{3} x_i y_i                                          \\
    \rel{g}_{\mu \nu} \rel{u}^\mu \rel{v}^\nu & = \sum_{\mu, \nu = 0}^{3} \rel{g}_{\mu \nu} \rel{u}^\mu \rel{v}^\nu
  \end{align}
}
Note that the greek index such as $\mu, \nu$ usually runs from $0$ to $3$, while the latin index such as $i, j, k$ usually runs from $1$ to $3$.

\subsection{Metric Tensor}
As per the convention in particle physics, we take the $(+, -, -, -)$ metric signature, and the Minkowski metric $\rel{\eta}_{\mu \nu}$ is denoted by:
\begin{align}
  \rel{\eta}_{\mu \nu}
  = \begin{pmatrix}
      \rel{\eta}_{00} & \rel{\eta}_{01} & \rel{\eta}_{02} & \rel{\eta}_{03} \\
      \rel{\eta}_{10} & \rel{\eta}_{11} & \rel{\eta}_{12} & \rel{\eta}_{13} \\
      \rel{\eta}_{20} & \rel{\eta}_{21} & \rel{\eta}_{22} & \rel{\eta}_{23} \\
      \rel{\eta}_{30} & \rel{\eta}_{31} & \rel{\eta}_{32} & \rel{\eta}_{33}
    \end{pmatrix}
  = \begin{pmatrix}
      1 & 0  & 0  & 0  \\
      0 & -1 & 0  & 0  \\
      0 & 0  & -1 & 0  \\
      0 & 0  & 0  & -1
    \end{pmatrix}
\end{align}

\subsection{Units}
In principle, we use SI units:
\begin{alignat}{2}
  \text{Length} & = & \bab{L} & = \si{\meter}    \\
  \text{Time}   & = & \bab{T} & = \si{\second}   \\
  \text{Mass}   & = & \bab{M} & = \si{\kilogram} \\
  \text{Charge} & = & \bab{C} & = \si{\coulomb}
\end{alignat}
However, for particle physics, we often use natural units, where:
\begin{align}
  \hbar = c = 1
\end{align}
and the unit of energy in electronvolts ($\si{\electronvolt}$) is used.